\documentclass[11pt]{article} % use larger type; default would be 10pt

\usepackage[utf8]{inputenc} % set input encoding (not needed with XeLaTeX)

\usepackage{graphicx}
\usepackage[czech]{babel}
\usepackage{amsfonts}
\usepackage{amsmath}
\usepackage{url}

\usepackage{bm}
\usepackage{geometry} 
\usepackage{subfigure}
\geometry{a4paper} 


\usepackage{listings}
\usepackage{ae}

\begin{document}
    \title{Dokumentace knihovny Canvas}
    \author{Karel Bílek}
    \date{10. května 2010}
    \maketitle
    
    SMAZAT MATRIX
\section{Úvod}
Knihovna Canvas je knihovna, určená pro kreslení jednoduchých geometrických objektů na obrazovku, jejich zobrazování a manipulaci. 

Implementováním podtříd pro některé třídy je pak možné jednoduše vytvářet nové geometrické objekty.

\section{Stručný popis}
Při kreslení canvasu postupuje knihovna takto:

\begin{itemize}
    \item Podívá se na seznam \uv{svých} geometrických objektů (třída \texttt{shape}), jestli v některém z nich bylo od posledního kreslení změna - pokud ne, vrátí totéž, co minule, bez kreslení. Pokud kreslí poprvé, musí nakreslit vše.
    \item Poté zjistí, na jakém prostoru se změny staly (pokud kreslím poprvé, tento prostor = celý rozměr canvasu).
    \item Poté na \textbf{každý} \texttt{shape} zavolá funkci \texttt{get\_pixels} (s tím, že řekne, na jakém úseku se kreslí). Stane se tam následující:
    \begin{itemize}
        \item Pokud byl objekt nakreslen a od té doby nezměněn, vrátí se jeho nakreslení
        \item Jinak nejdřív \texttt{shape} vezme svoje okraje (třída \texttt{curve}) a podle toho, jestli \texttt{curve} umí, nebo neumí vrátit sama sebe s určitou tloušťkou geometricky, se buď vrátí geometricky a vyplní, nebo se nakreslí bod po bodu
        \item A nakonec se vyplní samotný obsah - tak, že krajní \texttt{curves} vrátí svojí reprezentaci jako pole začátků a konců, poté se provede algoritmus, lépe popsaný v \cite{Pelikan}.
    \end{itemize}
\end{itemize}

Manipulace se samotnými \texttt{shapes} v \texttt{canvas} je následující:
\begin{itemize}
    \item Buď se může přímo zavolat funkce \texttt{rotate}, \texttt{resize} nebo \texttt{move}, které by měly zaručeně fungovat vždy.
    \item Nebo můžou mít \texttt{shape} implementované tzv. vlastnosti (properties)
    \begin{itemize}
        \item property Name má \texttt{shape} vždy, stejně jako barvy čáry a výplně a tloušťku čáry
        \item ostatní properties můžou být definovány v odpovídajícím \texttt{shape\_type} (o tom viz níže), například u kruhu je to střed a poloměr, u úsečky dvě čáry apod.
        \item \uv{můžu} píšu proto, protože uživatel nemusí chtít properties v rámci vlastní podtřídy \texttt{shape\_type} definovat - nastavování properties je totiž implementováno možná trochu netradičním způsobem.
    \end{itemize}
\end{itemize}
    
Zbytek programu jsou podtřídy \texttt{shape\_type} a \texttt{curve}.

Podtřída třídy \texttt{shape\_type} reprezentuje typ tvaru. Některé v knihovně jsou např. \texttt{elipse}, \texttt{segment} (úsečka) apod. Podtřída \texttt{shape\_type} musí mít definováno:
\begin{itemize}
    \item konstruktor, co vytvoří \texttt{curves}
    \item proceduru, kterou daný \uv{naklonuje} sám sebe
\end{itemize}
a dále nepovinně procedury na seznam všech možných properties a na změnu a hodnotu nějaké property.\footnote{Seznam všech properties by, logicky vzato, měla být funkce \emph{třídy} a ne instance, ale C++ statické virtuální funkce neumí.}

Podtřída třídy \texttt{curve} má mít:
\begin{itemize}
    \item proceduru na \uv{naklonování} sama sebe
    \item procedury na otočení, posunutí a zvětšení
    \item proceduru na vrácení všech segmentů křivky \uv{bod po bodu}
    \item proceduru, co vrací, jestli umím nebo ne nakreslit sám sebe s tloušťkou jako \texttt{shape\_type} - a pokud ano, tak jak přesně
\end{itemize}

V předchozí verzi uměla knihovna i antialias - nepodařilo se mi ho v konečné verzi zprovoznit, proto ho nazývám experimentální.

\section{Implementace knihovny}

Jak implementovat knihovnu ukazuje malý program, nazvaný prostě \texttt{qt}, který také ukazuje téměř všechny funkce knihovny.



\begin{thebibliography}{9}

   \bibitem{Pelikan}
     \url{http://cgg.mff.cuni.cz/~pepca/lectures/npgr003.html}
     
 \end{thebibliography}


\end{document}